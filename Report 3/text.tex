\documentclass{article}
\usepackage[greek,english]{babel}
\usepackage{alphabeta}
\usepackage{listings}
\usepackage{xcolor}
\usepackage[backend=biber]{biblatex}
\addbibresource{./bibl.bib}
\usepackage{hyperref}
\lstset{style=mystyle}
\usepackage{mathtools}
\usepackage{graphicx}
\usepackage{blindtext}
\usepackage{geometry}
\usepackage{listings}
\usepackage{amsmath}
\usepackage{amsfonts}
\usepackage{steinmetz}
\usepackage{algorithm}
\usepackage[noend]{algpseudocode}
\usepackage[shortlabels]{enumitem}
\usepackage{tikz}
\usepackage{fdsymbol}
% \newcommand{\comment}[1]{}
\renewcommand{\labelitemii}{\(\medblackdiamond\)}
\renewcommand{\labelitemiii}{\(\medblacksquare\)}
\renewcommand{\labelitemiv}{\(\medblackcircle\)}%
 \geometry{
 a4paper,
 total={170mm,257mm},
 left=20mm,
 top=20mm,
 }

\definecolor{codegreen}{rgb}{0,0.6,0}
\definecolor{codegray}{rgb}{0.5,0.5,0.5}
\definecolor{codepurple}{rgb}{0.58,0,0.82}
\definecolor{backcolour}{rgb}{0.95,0.95,0.92}

\lstdefinestyle{mystyle}{
    backgroundcolor=\color{backcolour},   
    commentstyle=\color{codegreen},
    keywordstyle=\color{magenta},
    numberstyle=\tiny\color{codegray},
    stringstyle=\color{codepurple},
    basicstyle=\ttfamily\footnotesize,
    breakatwhitespace=false,         
    breaklines=true,                 
    captionpos=b,                    
    keepspaces=true,                 
    numbers=left,                    
    numbersep=5pt,                  
    showspaces=false,                
    showstringspaces=false,
    showtabs=false,                  
    tabsize=2
}
\title{3η Σειρά Ασκήσεων, Αλγόριθμοι και Πολυπλοκότητα}
\author{Πέτρος Αυγερίνος 03115074}
\date{18 Δεκέμβρη, 2023}

\begin{document}
\maketitle
\tableofcontents
\pagebreak

\section{Άσκηση 1: Υπολογισιμότητα}
\begin{enumerate}
    \item{}
Αρχικά, γνωρίζουμε ότι ένας αλγόριθμος ημιαπόφασης διακόπτεται και επιστρέφει θετική απάντηση για
είσοδο ενός στιγμιοτύπου που ανήκει στο σύνολο λύσεων του προβλήματος, ενώ στην περίπτωση όπου το
στιγμιότυπο δεν ανήκει στο σύνολο των λύσεων ο αλγόριθμος δεν διακόπτεται ποτέ επιστρέφοντας αρνητική
απάντηση. Αν θεωρήσουμε τώρα πολυώνυμο $p(x,y_1,y_2,\dots,y_n) = 0$ όπου $x,y_1,y_2,\dots,y_n$
οι συντελεστές του πολυωνύμου της διοφαντικής εξίσωσης, υπολογίζοντας όλους τους πιθανούς συνδιασμούς
των συντελεστών, θα προκύπτουν όλες οι πιθανές λύσεις και μη του προβλήματος των διοφαντικών εξισώσεων.
Επομένως θα γνωρίζουμε τελικά για ποιους συντελεστές το πρόβλημα τερματίζει ή όχι. Η διατύπωση αυτή 
ισοδυναμεί με τον αλγόριθμο o οποίος τρέχει για πάντα και βρίσκει όλους εκείνους τους συντελεστές που δίνουν
λύση στο πρόβλημα των διοφαντικών εξισώσεων.\cite{mrdp}\\

Επομένως ο παραπάνω αλγόριθμος είναι ένας αλγόριθμος ημιαπόφασης και κάθε αλγόριθμος ημιαπόφασης
ανήκει στην κλάση RE, άρα το πρόβλημα των διοφαντικών εξισώσεων ανήκει στο RE.\\

    \item{}
Το πρόβλημα τερματισμού ορίζεται ως εξής: Δεδομένου ενός προγραμμάτος υπολογισμού, αποφάσισε αν το πρόγραμμα
τερματίζει ή όχι. Στο ερώτημα α) για την απόδειξη ότι το πρόβλημα των διοφαντικών εξισώσεων ανήκει στη 
κλάση RE κάναμε ουσιαστικά την αναγωγή του προβλήματος στο πρόβλημα τερματισμού.\\

Έστω H η συνάρτηση του προβλήματος τερματισμού, $DEP$ ο αλγόριθμος που ορίσαμε στο ερώτημα α) και $p$ οι 
συντελεστές του πολυωνύμου:\\\break
$H(DEP,p) = 
\begin{cases}
    \text{stops} &\text{if $DEP(p)$ stops}\\
    \text{doesn't stop} &\text{if $DEP(p)$ doesn't stop}
\end{cases}$\\\break

Επομένως μπορούμε να δούμε ότι για κάθε είσοδο μπορεί να αποδειχθεί αν το πρόβλημα του τερματισμού 
θα σταματήσει ή όχι, επομένως ανήκει στο RE-πλήρες.\\

    \item{}
Έστω η μηχανή αυτή Touring M ισοδυναμεί με τη $DEP$. Από τα προηγούμενα ερωτήματα είναι σαφές πως για
το πρόβλημα του τερματισμού με είσοδο τη $DEP$ θα υπάρχουν συντελεστές για τους οποίους το halting problem 
θα δίνει αρνητική απάντηση, επομένως το καθολικό πρόβλημα του τερματισμού εξ'ορισμού δεν μπορεί να δώσει
καθολική απάντηση για κάθε είσοδο. Επομένως ανήκει στη κλάση RE-hard.\\
\end{enumerate}
\pagebreak

\section{Άσκηση 2: Πολυπλοκότητα - Αναγωγές}
\begin{enumerate}
    \item{}
Έστω μία λογική έκφραση S σε κανονική συζευκτική μορφή η οποία δεν ικανοποιείται για καμία ανάθεση των μεταβλητών της.
Αν $C_i$ οι προτάσεις της εκφράσης έστω $k$ στον αριθμό, όπου μία θα πρέπει να είναι αντίφαση, από την
παραδοχή μας για την S, ορίζουμε γράφο:\\ \break
$G(V,E) : V=\{<a,i>\:a\in Ci \}$ και $E = \{(<a,i>,<b,j>):i \neq j ; b \neq a’\}$\\ \break

Θα πρέπει για το παρόν πρόβλημα μια κλίκα να δημιουργείται με την προυπόθεση ότι οι κόμβοι της 
επαληθεύουν τη έκφραση S.\\

Έτσι βλέπουμε ότι δημιουργούνται κλίκες τάξης ίσης με τον αριθμό των προτάσεων της έκφρασης, στις οποίες συμμετέχουν
κόμβοι ένας από κάθε πρόταση της έκφρασης, όμως δεν υπάρχει ανάθεση για να επαληθεύεται η S, επομένως 
δεν υπάρχει κλίκα με τουλάχιστον $k$ κόμβους.\\

    \item{}
Γνωρίζουμε ότι το Clique Decision Problem, το συμπλήρωμα του NoLargeClique, ανήκει στη κλάση NP αφού 
μια λύση του μπορεί να ελεγχθεί ως προς την ορθότητά της σε πολυωνυμικό χρόνο. Ας το αποδείξουμε:\\

Έστω S υποσύνολο γράφου G το οποίο περιέχει κόμβους μέσα σε μία κλίκα. Πρέπει να ελέγξουμε αν υπάρχει 
κλίκα μεγέθους k μέσα στο γράφο. Υπολογίζουμε αρχίκα αν ο αριθμός των κόμβων μέσα στο S ισούται με το k 
το οποίο κοστίζει σταθερό χρόνο. Ύστερα υπολογίζουμε για κάθε κόμβο μέσα στο S αν ο αριθμός των εξερχόμενων
ακμών ισούται με $k-1$ το οποίο κοστίζει $O(k^2)$. Στην χειρότερη περίπτωση το k θα ισούται με n, όπου n
ο συνολικός αριθμός των κόμβων στο G. Άρα η επαλήθευση της λύσης γίνεται σε πολυωνιμικό χρόνο.\\

Επίσης γνωρίζουμε πως μπορεί να γίνει αναγωγή από το SAT στο Clique Decision Problem όμοια με το ερώτημα α),
όμως το SAT είναι NP-hard, άρα και το Clique Decision Problem έιναι NP-hard. Επομένως προκύπτει ότι το Clique 
Decision Problem είναι NP-πλήρες αφού ανήκει και στις δύο κλάσεις NP και NP-hard. Με την χρήση της
δοθείσας ιδιότητας προκύπτει ότι το NoLargeClique είναι τελικά coNP-πλήρες.\\

    \item{}
Έστω πρόβλημα $P \in NP-complete$ και $P \in (NP \cap coNP) \Rightarrow P \in NP$ και $P \in coNP$.\\\break
Από τα προηγούμενα ερωτήματα γνωρίζουμε αν $P \in NP \Rightarrow \bar{P} \in coNP$ και $P \in coNP 
\Rightarrow \bar{P} \in NP$ άρα $\bar{P} \in (NP \cap coNP)$.\\\break

Υπάρχει επίσης πρόβλημα S το οποίο ανήκει στην κλάση NP το οποίο με αναγωγή πολυωνυμικού χρόνου μας 
δίνει το πρόβλημα P το οποίο είναι NP-πλήρες, αλλά και το πρόβλημα $\bar{P}$ το οποίο ανήκει στην κλάση
coNP-πλήρες.\\

Από τα παραπάνω το S ανήκει στην κλάση coNP. Αφού το S ανήκει και στις δύο αυτές κλάσεις τότε πρέπει 
$NP = coNP$.\\

\pagebreak
    \item{}
Για να δείξουμε ότι το πρόβλημα ανήκει στη κλάση NP-πλήρες αρχίκα θα πρέπει να δείξουμε ότι ανήκει στην
κλάση NP, και ύστερα ότι μια αναγωγή από ένα πρόβλημα NP-hard προκύπτει σε πολυωνυμικό χρόνο μας δίνει 
το αρχικό μας πρόβλημα.\\

Αρχικά μπορούμε να επαληθεύσουμε της αποδεκτότητα ενός πιστοποιητικού σε πολυωνυμικό χρόνο επομένως το NAE3SAT ανήκει στην κλάση NP.\\

Γνωρίζω ότι το 3-SAT problem είναι της κλάσης NP-hard επομένως αν δείξω ότι 3-SAT $\le_p NAE3SAT$ τότε 
θα προκύψει ότι NAE3SAT είναι NP-hard άρα και τελικά NP-πλήρες αφού ήδη δείξαμε πως είναι NP. Έστω λοιπόν 
μία έκφραση $S = \bigcap_{i=1}^{m}(l_1\cup l_2 \cup l_3)$ του προβλήματος 3-SAT. Αυτή μπορεί να 
μετασχηματιστεί σε μία έκφραση $S' = \bigcap_{i=1}^{m}(l_1 \cup l_2 \cup l_3 \cup s)$ του προβλήματος 
NAE4SAT, και ύστερα μπορούμε με αναγωγή να καταλήξουμε στο πρόβλημα του NAE3SAT όμοια με τη γενίκευση
της k-satisfiability του 3-SAT.\\

Μια ανάθεση που ικανοποιεί την S θα ικανοποιεί και την S' αν θέσουμε το $s=0$. Επίσης μια ανάθεση 
που ικανοποιεί την S' για $s=0$ θα πρέπει να έχει ένα έστω literal αληθή σε κάθε πρόταση και έτσι να
επαληθεύεται και η S. Αν στην S' θέσουμε $s=1$ τότε λόγω συμμετρίας θα μπορεί να αναστραφεί ώστε 
να προκύπτει μια ικανοποιητική ανάθεση με $s=0$. Άρα καταφέραμε με αναγωγή από το 3-SAT να φτάσουμε 
στο NAE3SAT, επομένως το πρόβλημα NAE3SAT είναι NP-πλήρες.\\

    \item{}
Θα κάνουμε αναγωγή του προβλήματος στο Vertex Cover για την εύρεση ελαχίστου καλλύματος κορυφών δημιουργώντας
ένα γράφο G τέτοιο ώστε για κάθε ζευγάρι ατόμων $u,v$ από ίδια κοινωνική ομάδα, θα έχουμε μία ακμή που ενώνει 
τα  $u,v$. Με την εύρεση του ελαχίστου καλλύματος κορυφών θα προκύψει ο αριθμός αντιπροσώπων, επομένως 
το ζητούμενο είναι δυνατό.\\
\end{enumerate}

\pagebreak

\section{Άσκηση 3: Προσεγγιστικοί Αλγόριθμοι - Vertex Cover - TSP}
\begin{enumerate}
    \item{}
\begin{enumerate}
    \item{}
        Θα ελέγξουμε όλους τους κόμβους πολλαπλές φορές αναζητώντας ακάλυπτες ακμές, στις οποίες 
        θα αναθέτουμε το μικρότερο δυνατό βάρος κάθε φορά, εισάγωντας την ακμή με το μικρότερο βάρος 
        μέσα στο κάλλυμα. Ο αλγόριθμος θα διακόψει μόλις το C είναι vertex cover, επομένως θα συμμέτεχουν
        όλες οι ακμές.\\
    \item{}
        Κάθε φορά που συμμετέχει ένας κόμβος σε μία ακμή για την δημιουργία του κυρτού καλλύματος, αφαιρείται
        από αυτόν ένα μέρος του βάρους του και τοποθετείται στην ακμή. Κάθε ακμή ενημερώνεται μόνο μία
        φορά και λαμβάνει ως κόστος το ελάχιστο βάρος των δύο κορυφών τηςμ το συνολικό βάρος που κατανέμεται
        στις ακμές είναι το πολύ διπλάσιο του συνολικού βάρους του κυρτού καλλύματος.\\
    \item{}
        Αυτό ισχύει επειδή σε μια βέλτιστη λύση, κάθε ακμή e πρέπει να καλύπτεται από τουλάχιστον μία κορυφή.
        Το βάρος αυτής της κορυφής θα είναι τουλάχιστον c(e), δεδομένου ότι c(e) ορίζεται ως το ελάχιστο βάρος
        των δύο κορυφών της ακμής. Έτσι, το συνολικό βάρος μιας βέλτιστης λύσης δεν μπορεί να είναι μικρότερο
        από το άθροισμα των c(e) για όλες τις ακμές.\\

        Το σχήμα που ζητείται είναι το εξής:\\ 

        \begin{center}
            \includegraphics[scale=0.2]{./graph.png}
        \end{center}\\

        Σε αυτή την περίπτωση ο αλγόριθμος επιλέγει τους κόμβους Α και Β για εισαγωγή στο κυρτό κάλλυμα, 
        και προσδίδει συνολικό κόστος ίσο με 4, ενώ η βέλτιστη επιλογή είναι ο κόμβος C με κόστος 2.\\

        

\end{enumerate}
    \item{}
        Θα μετατρέψουμε ένα πρόβλημα Hamilton Cycle σε TSP με τον εξής τρόπο, ορίζουμε το κόστος για κάθε 
        ακμή του Hamilton Cycle ίσο με τη μονάδα, και το κόστος των ακμών που δεν αναγράφονται στο Hamilton 
        Cycle ίσο με μία μεγάλη σταθερά M. Αν υπάρχει πολυωνυμικός αλγόριθμος με λόγο προσέγγισης για το TSP,
        τότε θα μπορούσαμε να εφαρμόσουμε αυτόν τον αλγόριθμο στο παραγόμενο στιγμιότυπο του TSP που προκύπτει 
        από την παραπάνω αναγωγή. Αν ο αλγόριθμος επιστρέφει λύση με συνολικό κόστος μικρότερο ή ίσο του μήκους
        της βέλτιστης διαδρομής (αν περιέχει k ακμές, το κόστος θα έιναι ίσο με k) θα μπορούσαμε να χρησιμοποιήσουμε τη λύση αυτή
        τη λύση για να κρίνουμε αν υπάρχει ή όχι κύκλος Hamilton στον αρχικό γράφο.\\

        Αν υπάρχει κύκλος Hamilton τότε στο παραγόμενο TSP το μήκος της βέλτιστης διαδρομής θα είναι ίσο
        με τον αριθμό των κορυφών του γράφου. Αν δεν υπάρχει, τότε η βέλτιστη διαδρομή θα έχει τουλάχιστον 
        μία ακμή με κόστος Μ, επομένως το συνολικό κόστος που θα προκύψει θα είναι μεγαλύτερο από το k.\\

        Αν ο αλγόριθμος λοιπόν επιστρέψει λύση με συνολικό κόστος $\le k$ τότε μπορούμε να διακρίνουμε αν
        υπάρχει ή όχι ένας κύκλος Hamilton στον αρχικό γράφο. Αυτό ισοδυναμεί με τη δυνατότητα λύσης NP-complete
        προβλήματος σε πολυωνυμικό χρόνο, το οποίο σημαίνει P=NP.\\
\end{enumerate}

\pagebreak
\printbibliography


\end{document}
