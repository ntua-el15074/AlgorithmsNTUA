\documentclass{article} \usepackage[greek,english]{babel}
\usepackage{alphabeta}
\usepackage{listings}
\usepackage{xcolor}

\lstset{style=mystyle}
\usepackage{mathtools}
\usepackage{graphicx}
\usepackage{blindtext}
\usepackage{geometry}
\usepackage{listings}
\usepackage{amsmath}
\usepackage{amsfonts}
\usepackage{steinmetz}
\usepackage{algorithm}
\usepackage[noend]{algpseudocode}
\usepackage[shortlabels]{enumitem}
\usepackage{tikz}
\usepackage{fdsymbol}
% \newcommand{\comment}[1]{}
\renewcommand{\labelitemii}{\(\medblackdiamond\)}
\renewcommand{\labelitemiii}{\(\medblacksquare\)}
\renewcommand{\labelitemiv}{\(\medblackcircle\)}%
 \geometry{
 a4paper,
 total={170mm,257mm},
 left=20mm,
 top=20mm,
 }

\definecolor{codegreen}{rgb}{0,0.6,0}
\definecolor{codegray}{rgb}{0.5,0.5,0.5}
\definecolor{codepurple}{rgb}{0.58,0,0.82}
\definecolor{backcolour}{rgb}{0.95,0.95,0.92}

\lstdefinestyle{mystyle}{
    backgroundcolor=\color{backcolour},   
    commentstyle=\color{codegreen},
    keywordstyle=\color{magenta},
    numberstyle=\tiny\color{codegray},
    stringstyle=\color{codepurple},
    basicstyle=\ttfamily\footnotesize,
    breakatwhitespace=false,         
    breaklines=true,                 
    captionpos=b,                    
    keepspaces=true,                 
    numbers=left,                    
    numbersep=5pt,                  
    showspaces=false,                
    showstringspaces=false,
    showtabs=false,                  
    tabsize=2
}
\title{1η Σειρά Ασκήσεων, Αλγόριθμοι και Πολυπλοκότητα}
\author{Πέτρος Αυγερίνος 03115074}
\date{7 Νοεμβρίου, 2023}

\begin{document}
\maketitle
\tableofcontents
\pagebreak

\section{Πλησιέστερο Ζεύγος Σημείων}
    α. Για την εύρεση του πλησιέστερου ζέυγους σημείων στον τρισδιάστατο χώρο
    θα εργαστούμε παρόμοια με τις δύο διαστάσεις. Αρχικά θα ταξινομήσουμε κατά $x,y,z$ με
    κόστος $O(3nlog(n))$. Αργότερα μπορούμε να βρούμε μια επιφάνεια η οποία χωρίζει
    τον τρισδιάστατο χώρο σε δύο set $P,Q$ μέσω του median ενός από τις τρείς διαστάσεις
    για τα σημεία αυτά, έστω $x_{median}$. \\

    Έστω η επιφάνεια αυτή $H$ και αναδρομικά βρίσκουμε $\delta_{P}, \delta_{Q}$,
    τις μικρότερες αποστάσεις μέσα σε κάθε υποπρόβλημα.Από τον αλγόριθμο των δύο διαστάσεων είδαμε
    ότι μπορούμε με πεπερασμένο αριθμό συγκρίσεων για ένα σημείο να βρούμε όλα τα πιθανά πλησιέστερα σημεία με 
    $d(p_i,p_j) \le \delta, \forall i \neq j$ με την χρήση ενός τετραγώνου διαστάσεων $(2 \cdot \delta)^2$.
    Στην περίπτωση του τρισδιάστατου χώρου μπορούμε παίρνοντας το $\delta = min(\delta_{P},\delta_{Q})$
    να κατασκευάσουμε ένα τρισδιάστατο slab με μήκος $2\delta$ γύρω από το $H$.  \\

    Χωρίζοντας το παραπάνω slab σε κύβους με όγκο $(2\cdot \delta)^3$, κατά το μέγιστο μπορούμε μέσα σε αυτόν τον
    κύβο να έχουμε 9 σημεία σε απόσταση $\delta$ από το $H$ που ανήκουν στο $P$, 9 σημεία πάνω στο $Η$ τα οποία ανήκουν
    στο $P$, 9 σημεία σε απόσταση $\delta$ από το $H$ που ανήκουν στο $Q$ και 9 σημεία πάνω στο $Η$ τα οποία ανήκουν στο $Q$.
    Επομένως για ένα σημείο $p \in P$ θα πρέπει να γίνουν συνολικά 35 έλεγχοι. για κάθε διάσταση $y,z$ άρα σύνολο 70 έλεγχοι,
    καθώς με τις ταξινομήσεις έχουμε ουσιαστικά προβάλει τα σημεία μας πάνω στο $Η$. Άρα στην χειρότερη περίπτωση όπου το slab
    εμπερίεχει όλα τα στοιχεία, $\frac{n}{2}$ σε κάθε set $P,Q$ η συνολική πολυπλοκότητα για το κομμάτι του βασίλευε προκύπτει $Ο(70 \cdot n)$.
    Μπορούμε επίσης να μειώσουμε τις συγκρίσεις αρκεί να ανατρέξουμε μόνο ένα από τα $P,Q$ με τα αντίστοιχα της άλλης μεριάς του slab. Αυτό όμως
    δεν έχει πραγματικό αντίκτυπο στην πολυπλοκότητά του αλγορίθμου. \\

    \begin{center}
        \includegraphics[scale=0.5]{./closest_pair_3d_cube.png}

    \end{center}

    Η συνολική πολυπλοκότητα του αλγορίθμου μας προκύπτει $O(n \cdot log(n)^3)$. Από Master Theorem 
    προκύπτει $T(n) = 2\cdot T(\frac{n}{2}) + O(n) = O(nlog(n))$.
    \begin{algorithm}
        \caption{Closest\_Pair\_3D($P= [ p_i(x_i,y_i,z_i)\in 3D]:i \in[0,n], n$)}
        \begin{algorithmic}[1]
            \State Sort by each dimension and create arrays $X_s, Y_s, Z_s$ \Comment{$O(n\cdot log(n)^3$)}
            \State \delta = Find\_Closest\_Distance$(X_s, Y_s, Z_s, n)$
            \State \textbf{return} \delta

        \end{algorithmic}
    \end{algorithm}

    \begin{algorithm}
        \caption{Find\_Closest\_Distance$(X_s, Y_s, Z_s, n)$}
        \begin{algorithmic}[1]
            \If{$n == 2$}
                \Return $d(p1,p2)$
            \EndIf
            \If{$n == 3$}
                \Return $min(d_1, d_2, d_3)$
            \Else
                \State $x_{median} = X_s[n/2]$
                \State $X_p = X_s[1:n/2]$, $X_q = X_s[n/2+1:n]$
                \State $k = 0$
                \For{$i \in [0,n]$}
                    \If{$Y_s.x_i \le x_{median}$}
                        \State $Y_p.insert(P.y_i)$
                    \Else
                        \State $Y_q.insert(P.y_{i})$
                    \EndIf
                    \If{$Z_s.x_i \le x_{median}$}
                        \State $Z_p.insert(P.z_i)$
                    \Else
                        \State $Z_q.insert(P.z_{i})$
                    \EndIf
                \EndFor
                \State $\delta_p = Find\_Closest\_Distance(X_p, Y_p, Z_p, n/2)$
                \State $\delta_q = Find\_Closest\_Distance(X_q, Y_q, Z_q, n/2)$
                \State $\delta = min(\delta_p, \delta_q)$ \Comment{$2\cdot T(\frac{n}{2})$}
                \State Construct $S'$ containing all points that are in distance $\delta$ from H plane.(assume k points) \Comment{$O(n)$}
                \For{$i \in [0,k]$}
                    \For{$j \in [i+1,i+35]$} 
                        \If{$j \le k$}
                            \If{$d(p_i,p_j) < \delta$}
                            $\delta = d(p_i,p_j)$ \Comment{Finite, as proven above}
                            \EndIf
                        \EndIf
                    \EndFor
                \EndFor
                \Return $\delta$

            \EndIf

        \end{algorithmic}
    \end{algorithm}
\pagebreak

β. Γνωρίζουμε ότι\\
\begin{center}
    $\delta^* \in[l, c\cdot l] \Rightarrow l \le \delta^* \le c \cdot l$
\end{center}\\ 
όπου θα υπάρχουν δύο σημεία $p_{1}^*,p_{2}^*$ για τα οποία θα ισχύει $d(p_{1}^*,p_{2}^*) = \delta^*$. Άρα: \\
\begin{center}$l \le d(p_{1}^*,p_{2}^*) \le cl \Rightarrow \\
    l \le \sqrt{(x_1-x_2)^2+(y_1-y_2)^2+ \dots + (d_1-d_2)^2} \le cl \Rightarrow \\
    1 \le \sqrt{(\frac{x_1-x_2}{l})^2 + (\frac{y_1-y_2}{l})^2 + \dots + (\frac{d_1-d_2}{l})^2} \le c \Rightarrow \\ \\ 
    1 \le d(p_{1}^l,p_{2}^l) \le c$.
\end{center} \\
Αν $\forall p_i$ κανονικοποιήσουμε με τον παραπάνω τρόπο μπορούμε να δημιουργήσουμε ένα grid στο οποίο για τα σημεία μας
θα ισχύει $1 \le d(p_{i}^*,p_{j}^*) \forall i,j$ με $i \neq j$. \\

\begin{center}
    \includegraphics[scale=0.5]{./closest_pair_D.png}
\end{center}
Έστω $c$ μια ποσότητα ανάλογη της ποσότητας συγκρίσεων που θα κάνουμε για το παραπάνω grid για ένα σημείο $p_i$.
Μπορούμε να εργαστούμε παρόμοια με το προηγούμενο ζητούμενο όπου ελέγχουμε υπερκύβους πλευράς $2\cdot c$.
Για δύο διαστάσεις είχαμε $(2\cdot c)^2$, για τρεις $(2\cdot c)^3$, για έναν υπερκύβο $d$ διαστάσεων 
θα έχουμε τελικά $(2\cdot c)^d$. Επομένως προκύπτει πεπερασμένος αριθμός συγκρίσεων για κάθε σημείο 
με πολυπλοκότητα $O((2\cdot c)^d \cdot n)$. Έτσι μπορούμε σε γραμμικό χρόνο να βρούμε το πιο κοντινό
ζευγάρι σημείων.

    \begin{algorithm}
        \caption{Closest\_Pair\_DD(P,n)}
        \begin{algorithmic}[1]
            \State $\delta \to \infty$
            \For{each dimension in a single $P_i$} \Comment{$O(d\cdot n)$}
                \For{$i \in [0,n]$}
                    \State Normalise $p_i$ by $l$ for said dimension
                \EndFor
            \EndFor
            \For{$i \in [0,n]$}
            \State Calculate $(2\cdot c)^d$ distances for $p_i$ \Comment{$O((2\cdot c)^d\cdot n)$}
                \If{$d(p_i,p_j) < \delta$}
                    \State $\delta = d(p_i,p_j)$
                \EndIf
            \EndFor
        \Return $\delta$

        \end{algorithmic}
    \end{algorithm} \\

\pagebreak


\section{Πόρτες Ασφαλείας στο Κάστρο}
Ουσιαστικά αντιλαμβανόμαστε ότι αναζητούμε μια binary συμβολοσειρά $n$ ψηφίων με την οποία θα είναι
εφικτό να είναι ανοιχτές όλες οι πόρτες ταυτόχρονα ώστε να περάσουμε. Αν δοκιμάζαμε κάθε δυνατή επιλογή
θα προέκυπταν $2^n$ συνδυασμοί το οποίο δεν είναι υπολογιστικά συνετό. Επομένως θα πρέπει να εργαστούμε διαφορετικά.\\
Αφού η μόνη γνώση που έχουμε στη διάθεση μας για τις διάφορες συμβολοσειρές που μπορούν να προκύψουν είναι για την τελευταία πόρτα
που μένει κλειστή θα πέπει σειριακά για κάθε πόρτα να ψάξουμε την τιμή και τον διακοπτή για τον οποίο
η πόρτα αυτή ανοίγει. \\
Για την εύρεση της πρώτης πόρτας, αρχικά θα οδηγήσουμε όλους τους διακόπτες σε μία τιμή, 0 ή 1, και
αν η πόρτα ανοίξει με την κίνησή μας αυτή τότε γνωρίζουμε ότι ανοίγει με την τιμή που θέσαμε, αλλιώς
με την άλλη τιμή. Πλέον λοιπόν γνωρίζουμε για ποια τιμή ανοίγει η πρώτη πόρτα. Αρκεί τώρα να βρούμε τον
διακόπτη για τον οποίο συμβαίνει αυτό. \\
Για την εύρεση του διακόπτη ανοίγουμε $\frac{n}{2}$ διακόπτες και κλείνουμε τους υπόλοιπους. Αφού γνωρίζουμε
για ποια τιμή ανοίγει η πόρτα με την διαμέριση των διακοπτών αυτή μπορούμε να αντιληφθούμε σε ποιό 
από τα δύο μισά θα πρέπει να εργαστούμε. Αν η πόρτα ανοίξει για τα μισά που θέσαμε στη τιμή για την 
οποία γνωρίζουμε ότι ανοίγει η πόρτα, τότε θα ψάξουμε αναδρομικά μέσα σε αυτό το σύνολο. Στην περίπτωση
όπου κάνουμε λάθος ψάχνουμε το διπλανό σύνολο το οποίο έχει κόστος σταθερό. Επομένως με αυτή την αναδρομή
προκύπτει κόστος για την εύρεση του διακόπτη της πρώτης πόρτας $\Theta(log(n))$. \\
Διατηρούμε τον διακόπτη της πρώτης πόρτας στη θέση που πρέπει για να ανοίγουν και προχωράμε αναδρομικά 
για την εύρεση της τιμή για την οποία ανοίγει η δεύτερη πόρτα, με την οδήγηση όλων των υπολοιπόμενων διακοπτών
σε μία από τις τιμές 0 ή 1. Μόλις βρεθεί η τιμή μπορούμε πάλι με κόστος αυτή τη φορά $\Theta(log(n-1))$
να βρούμε ποιος διακόπτης είναι υπεύθυνος για την δεύτερη πόρτα. Αναδρομικά κάνω το ίδιο για την i-ιοστή πόρτα. \\ \break
Συνολικά λοιπόν προκύπτει για τις συνολικές διαμορφώσεις: $\Delta(n) = \sum_{i=1}^{n}\Theta(log(n)-i+1)
= \Theta(n\cdot log(n))$
    \begin{algorithm}
        \caption{Castle\_Doors(Doors,Switches)}
        \begin{algorithmic}[1]
            \For{each Door in Doors} \Comment{$\Theta(n)$}
            \State Switch all Switches on (1) \Comment{$\Theta(1)$}
                \If{Door opens}
                    \State Door.opens\_at \leftarrow 1
                \Else
                    \State Door.opens\_at \leftarrow 0
                \EndIf
                \State Split the Switches in $\frac{n}{2}$ parts each
                \State Recursively check each part, given where the door opens \Comment{$\Theta(log(n))$}
                \State Save the position of the Switch that opened the Door

            \EndFor

        \end{algorithmic}
    \end{algorithm}

\pagebreak

\section{Κρυμμένος Θησαυρός}
Αυτό το πρόβλημα είναι γνωστό και ως "Lost Cow Problem". \\ \break
Ο τρόπος με τον οποίο θα λύσουμε το παρόν πρόβλημα είναι ο διπλασιασμός της απόστασης που διανύουμε από
το $0$ κάθε φορά, μία από τα δέξια και μία από τα αρίστερα σε σχηματισμό "zig zag". 
Ονομάζουμε την απόσταση που διανύω στην επανάληψη $i$ του αλγορίθμου ως $r_{i} = 2^{i}$ όπου $i=\{1,2,3, \dots\}$
(Για το $i=0$ καμία βάση $(a^0 = 1)$ δεν μπορεί να αυξήσει την πληροφορία σχετικά με την εύρεση του θησαυρού, και δεν πρόκειται
να βρεθεί στο διάστημα μέχρι το $1$ και γι'αυτο δεν το συμπεριλαμβάνουμε στις επαναλήψεις μας). \\
'Άρα
είναι σαφές πως η απόσταση που θα διανύσω μέχρι να επιστρέψω στο $0$ για εκείνη την επανάληψη θα είναι $2\cdot r_i$.
Για τα $imod2 = 1$ ακολουθώ το δεξί μονοπάτι και για τα $imod2 = 0$ το αριστερό. Έστω ότι βρίσκω τον
θησαυρό σε απόσταση $|x|$ από το $0$ στην $n+2$ επανάληψη του αλγορίθμου. Επομένως η συνολική απόσταση
που θα έχω διανύσει θα είναι: \\ 
\begin{center}
    $D(n) = 2 + 2 + 4 + 4 + \dots + r_{n+1} + r_{n+1} + |x| =\\ \\
    2 \cdot \sum_{i=1}^{n+1}2^{i} + |x| = \\
    2 \cdot (2^{n+2}-2)+|x|$
\end{center} \\

Όμως αφού ο θησαυρός βρέθηκε στην επανάληψη $n+2$, αυτό σημαίνει ότι η εύρεσή του διέκοψε τον αλγόριθμο
σε κάποιο σημείο μεταξύ των αποστάσεων $2^n$ και $2^{n+2}$, επομένως προκύπτει ότι $|x| = 2^n + d'$. Άρα: \\

\begin{center}
    $D(n) = 
    2 \cdot (2^{n+2}-2)+|x| = \\
    2 \cdot (2^{n+2}-2)+2^n+d' = \\
    2 \cdot (4 \cdot 2^n -2) + 2^n +d' = \\
    9 \cdot 2^n -4 + d' \le 9 \cdot 2^n + d' \le 9 \cdot (2^n + d')  \Rightarrow \\
    9 \cdot (2^n + d')\le 9 \cdot |x|
    $
\end{center} \\

Βρήκαμε λοιπόν μία σταθερά $c=9$ για την οποία ο αλγόριθμός μας έχει πολυπλοκότητα $Ο(|x|)$.\\
Για την επιλογή της βάσης της εκθετικής συνάρτησης βημάτων, θα αποδείξουμε ότι βέλτιστη βάση 
$a \in \mathbb{N}$ είναι
για $a=2$. \\ Έστω ότι η βάση αυτή είναι $a$. Επομένως το $D(a,n)$ για ένα $n >> a$
θα είναι της μορφής:\\
\begin{center}
    $D(a,n) = 2\cdot \sum_{i=1}^{n+1}{a^i} + |x| = 2\cdot \sum_{i=1}^{n+1}{a^i} + a^n + d' = $\\
    $2 \cdot (\dfrac{a^{n+2}-1}{a-1}-1) + a^n + d' \xRightarrow{n>>a} 2 \cdot a^{n+2} + a^n + d' =$ \\
    $(2a^2 + 1)\cdot a^n + d'$ \\
\end{center}

Η οποία είναι μια γνησίως αύξουσα συνάρτηση ως προς $a$ με ελάχιστο ακέραιο δυνατό προς επιλογή $a=2$ καθώς,
για $a=1$ θα κάναμε ταλάντωση γύρω από το μηδέν με μήκος 1.
    \begin{algorithm}
        \caption{Hidden\_Treasure(X)}
        \begin{algorithmic}[1]
            \State $i \leftarrow 1$
            \While{Treasure is not found}
            \State Travel $2^i$ steps towards $Path_{imod2}$
            \If{Treasure is Found while travelling}
            \Return $i$ \Comment{$O(|x|$)}
            \Else
                \State Return to $0$
                \State $i \leftarrow i + 1$
            \EndIf
            \EndWhile

        \end{algorithmic}
    \end{algorithm}


\pagebreak

\section{Μη Επικαλυπτόμενα Διαστήματα}
α. Για την περίπτωση όπου χρησιμοποιούμε ως κριτήριο το μέτρο $|f_i - s_i|$ προκύπτει η εξής αναδρομική
σχέση: \\
\begin{center}
$C^*(A) = max\{|f_i-s_i|+C^*(A_i)\}, \text{ όπου }  A_i=\{j\in A: s_j \ge f_i\}$
\end{center} \\

Ταξινομώντας λοιπόν τα $|f_i - s_i|$ σε φθήνουσα σειρά προκύπτει ότι: \\
\begin{center}
    $|f_1 - s_1| > |f_2 - s_2| > \dots > |f_n - s_n| \text{ όπου } n \text{ το σύνολο των διαστημάτων}$ \\
\end{center}

Προκύπτει λοιπόν ο εξής αλγόριθμος: \\

    \begin{algorithm}
        \caption{$Sequences\_1(S)$}
        \begin{algorithmic}[1]
        \State Sort A via criterion $|f_i - s_i|$(max first)
        \State $j \leftarrow 1$
        \State $C \leftarrow \{1\}$ 
        \For{$i \leftarrow 2$ to $n$}
            \If{$s_i \ge f_j$}
                \State $C \leftarrow C \cup \{i\}$
                \State $j \leftarrow i$
            \EndIf
        \EndFor
        \Return $C$

        \end{algorithmic}
    \end{algorithm}

Ας λάβουμε υπόψιν όμως το εξής παράδειγμα.

\begin{center}
    \includegraphics[scale=0.5]{./sequences.png}
\end{center}

Ταξινομώντας τα 1,2,3 κατά φθήνουσα σειρά μέτρου, παίρνουμε πρώτο το μεγαλύτερο, στην πέριπτωσή μας το
3, το οποίο όμως είναι τελευταίο, επομένως ο αλγόριθμος θα τερματήσει χωρίς να συμπεριλάβει το 1 και 2 
στην απόφασή του λόγω του ελέγχου $s_i \ge f_j$. Είναι φανερό όμως πως η βέλτιστη λύση δεν είναι μόνο το
$\{3\}$ αλλά το $\{1,2,3\}$. Άρα είναι σαφές πως με αυτό το κριτήριο ο αγλόριθμός μας αποτυγχάνει. \\

Ταξινομώντας τώρα με βάση τα $f_i$ σε αύξουσα σειρά, την σειρά δηλαδή με την οποία οι διεργασίες ολοκληρώνονται
πρώτες κάθε φορά σε κάθε βήμα του αλγορίθμου, θα πάρουμε την βέλτιστη λύση η οποία είναι το σύνολο
$\{1,2,3\}$. Ο αγλόριθμος λοιπόν παίρνει την εξής μορφή που είδαμε και στο μάθημα. \\

    \begin{algorithm}
        \caption{$Sequences\_2(S)$}
        \begin{algorithmic}[1]
        \State Sort A via criterion $f_i$(min first)
        \State $j \leftarrow 1$
        \State $C \leftarrow \{1\}$ 
        \For{$i \leftarrow 2$ to $n$}
            \If{$s_i \ge f_j$}
                \State $C \leftarrow C \cup \{i\}$
                \State $j \leftarrow i$
            \EndIf
        \EndFor
        \Return $C$

        \end{algorithmic}
    \end{algorithm}

Και στις δύο περιπτώσεις λόγω ταξινόμησης θα προκύπτει πολυπλοκότητα $O(n\cdot log(n))$ λόγω
ταξινόμησης.

\pagebreak

β. Αρχικά θα ορίσουμε κάποια notations για το πρόβλημα ώστε να μπορούμε καλύτερα να το εκφράσουμε.
Θέτω $\Delta(n)$ το σύνολο όλων των διαστημάτων για το πρόβλημα μου, όπου $\delta_i \in \Delta(n)$ είναι τα 
διαστήματα αυτά. \\
Το $\delta_i$ εμπεριέχει πληροφορία σχετικά με την αρχή του sequence $s_i$, το τέλος $f_i$, το μέτρο
$|f_i - s_i|$, αν χρειαστεί θα το επεκτείνουμε για τις ανάγκες του προβλήματός μας. \\

\begin{lstlisting}[language=C++]
    struct { 
        int s; \\ start
        int f; \\ finish
        int m = f-s; \\ |f-s|
    } sequence;
\end{lstlisting} \\

Ταξινομώ το σύνολο $\Delta$ κατά αύξουσα σειρά των $f_i$. Πλέον ο άπληστος αλγόριθμος δεν θα καταφέρει
να μας δώσει τη λύση στο πρόβλημα, θα πρέπει να εργαστούμε με δυναμικό προγραμματισμό. Θα επιχειρήσουμε
να κατασκευάσουμε μία αναδρομική σχέση η οποία για κάθε βήμα του αλγορίθμου θα εντάσει στο πρόβλημα
ένα ακόμη διάστημα. Κάθε φορά που εντάσσουμε ένα διάστημα στο πρόβλημά μας θα επιλέγουμε την καλύτερη 
μεταξύ δύο περιπτώσεων, την περίπτωση αυτή δηλαδή που μεγιστοποιεί την συνάρτηση κόστους για το άθροισμα
μη επικαλυπτώμενων διαστημάτων.\\
\begin{enumerate}
    \item{Αν το $\delta_i$ επιλέγεται στην λύση για $i$ διαστήματα.}
    \item{Αν το $\delta_i$ δεν επιλέγεται στην λύση για $i$ διαστήματα.}
\end{enumerate}

Είναι σαφές πως για να διατηρώ την γνώση σχετικά με το πόσα διαστήματα έχουν εισαχθεί στο βήμα $i$
και το συνολικό βάρος της βέλτιστης λύσης μέχρι εκείνο το σημείο θα πρέπει να αποθηκεύσω τη γνώση 
αυτή σε ένα πίνακα. Έστω ο πίνακας αυτός $A[n]$ ο οποίος σε κάθε θέση $i$ περιέχει τη βέλτιστη λύση 
για την επιλογή διαστημάτων μέχρι εκείνο το σημείο. Θα χρειαστεί σε αυτό το σημείο μία προεργασία 
για την έυρεση των προηγούμενων μη επικαλυπτώμενων διαστημάτων $\forall \delta_i \in \Delta$ ώστε 
να μπορούμε να υλοποιήσουμε την αναδρομική σχέση που βρήκαμε παραπάνω. Για την εύρεση αυτών μπορούμε
να κάνουμε δυαδική αναζήτηση για το $s_i$ (την αρχή του διαστήματος για το οποίο ψάχνουμε τον αμέσως
προηγόυμενό του) μέσα στον ταξινομημένο κατά $f_i$ πίνακα $\Delta$ και να πάρουμε το αμέσως μικρότερο στοιχείο 
από το $s_i$. Η πολυπλοκότητα αυτής της διαδικασίας είναι σαφώς $Ο(log(n))$, και για όλα τα $n$ 
διαστήματα προκύπτει τελικά $O(n \cdot log(n))$. Μπορώ να αποθηκεύσω τον προηγούμενο αυτό μέσα στο ίδιο
struct με πριν, επομένως το struct μου θα πάρει την εξής μορφή: \\

\begin{lstlisting}[language=C++]
    struct { 
        int s; \\ start
        int f; \\ finish
        int m = f-s; \\ |f-s|
        int p; \\ index of previous
    } sequence;
\end{lstlisting} \\

Με την παραπάνω διαδικασία μπορούμε τώρα να γράψουμε την αναδρομική σχέση του αλγορίθμου μας. \\ 

\begin{center}
    $A[i] =
    \begin{cases}
        0 & i = 0 \\
        max\{\delta_i.m + A[\delta_i.p], Α[i-1]\} & i \neq 0
    \end{cases}$ \\ 
\end{center}

Με την παραπάνω αναδρομική σχέση γνωρίζουμε πως το τελευταίο στοιχείο του πίνακα $Α$ θα περιέχει την
βέλτιστη λύση για την εισαγωγή όλων των διαστημάτων του $\Delta$. Επομένως ο αλγόριθμός μας παίρνει την
εξής μορφή. \\ 

    \begin{algorithm}
        \caption{$Optimal\_Length(\Delta)$ \Comment{$O(n\cdot log(n)$}}
        \begin{algorithmic}[1]
            \State $A[0] = 0$
            \State Sort $\Delta$ by $f_i$ in ascending order \Comment{$O(n\cdot log(n))$}
            \For{$\delta \in \Delta$} \Comment{$O(n)$}
            \State Search for immediate previous $f$ of $\delta.s$ \Comment{$O(log(n))$}
            \EndFor
            \For{$i \in [1,n]$}
            $A[i] = max(s_i.m + A[s_i.p], A[i-1])$ \Comment{$O(n)$}
            \EndFor
            \Return $A[n]$
        \end{algorithmic}
    \end{algorithm}

\pagebreak

Καταφέραμε λοιπόν να βρούμε σε χρόνο $Ο(n\cdot log(n))$ το μέγιστο δυνατό μήκος για τα μη επικαλυπτώμενα
διαστήματα. Θα αναζητήσουμε τώρα το σύνολο $C$ των διαστημάτων του $\Delta$ για τα οποία ισχύει η
βέλτιστη αυτή λύση. \\
Ξεκινώντας από το τέλος του πίνακα $Α$ για $i=n$ και αφαιρώντας ένα ένα τα διαστήματα μπορούμε, κάθε φορά που
$A[i-1] < A[i]$ να συμπεράνουμε ότι το τελευταίο που αφαιρέσαμε συμμετέχει στην βέλτιστη λύση και έτσι
να το τοποθετήσουμε στο σύνολο $C$. Όμως θα πρέπει να λάβουμε υπόψιν και το αμέσως προηγούμενο του
διαστήματος $\delta_i$ καθώς η μείωση αυτή μπορεί να προκύψει και από αλλαγή συνόλου διαστήματων. Οπότε
κάθε φορά που παρατηρείται μείωση, θα ελέγχουμε αν υπάρχει προηγούμενο μη επικαλυπτώμενο και αν υπάρχει
θα συνεχίζουμε την αναζήτηση σε αυτό το προηγούμενο γλυτώνοντας ετσί κάποιους ελέγχους. \\

    \begin{algorithm}
        \caption{$Optimal\_Solution(n)$ \Comment{$O(n)$}}
        \begin{algorithmic}[1]
            \State $i=n$
            \If{$i=0$}
                \Return $C$
            \EndIf
            \If{$s_i.m + A[s_i.p] > A[i-1]$}
                \State Append $i$ to $C$
                \State $i=i-1$
                \State $Optimal\_Solution( s_i.p)$
            \Else
                \State $Optimal\_Solution( i)$
            \EndIf
            \Return $C$
        \end{algorithmic}
    \end{algorithm}







\pagebreak

\section{Παραλαβή Πακέτων}
1. Για την επίλυση του παρόντος προβλήματος θα κάνουμε χρήση άπληστου αλγόριθμου με κριτήριο τον
λόγο $\dfrac{w_i}{p_i}$. Ταξινομώντας λοιπόν $\forall i \in A$ τα $\dfrac{w_i}{p_i}$ σε φθήνουσα σειρά
μπορούμε κάθε φορά να εξυπηρετούμε τον πελάτη εκεινό του οποίο ο λόγος σημαντικότητας προς χρόνο
εξυπηρέτησης είναι μεγαλύτερος. Προκύπτει λοιπόν η εξής αναδρομική σχέση με την οποία θα
εργαστούμε. \\ \break
\begin{center}
    $G(A)=(\text{Εξυπηρέτηση Πελάτη $i$}) : max\{\dfrac{w_i}{p_i}\forall i \in A\}+G(A'=A-\{i\})$ \\
\end{center}

Ας υποθέσουμε τώρα ότι η ταξινόμησή μας γίνεται ως εξής:
$\dfrac{w_1}{p_1} \ge \dfrac{w_2}{p_2} \ge \dots \ge \dfrac{w_n}{p_n}$. 
Άρα η λύση του άπληστου αλγορίθμου $S$ θα είναι: \\
\begin{center}
    $S=w_1 \cdot p_1 + w_2 \cdot (p_1 + p_2) + \dots + w_n \cdot \sum_1^n{p_i}$ \\
\end{center} 

Έστω ότι υπάρχει μία λύση η οποία είναι η βέλτιστη για την οποία μεταθέτω ένα $w_i$ με το
$w_{i+1}$ όπου $i \in (1,n)$ και $\dfrac{w_i}{p_i} > \dfrac{w_{i+1}}{p_{i+1}}$. 
Έτσι προκύπτει ένα $S'$ το οποίο θα είναι ως εξής: \\
\begin{center}
    $S'= w_1 \cdot p_1 + w_2 \cdot (p_2 + p_2) + \dots + w_{i+1}\cdot \sum_1^{i-1}{p_k} + w_{i+1}\cdot
    p_{i+1}+ w_{i}\cdot \sum_1^{i+1}{p_k} + \dots  + w_n \cdot \sum_1^n{p_i}$
\end{center} \\
Για τα $S,S'$ μπορούμε να εργαστούμε ως εξής για να δούμε την ορθότητα του άπληστου αλγορίθμου: \\
Το κλάσμα $\dfrac{S}{S'} > 1$ αφού έχουμε υποθέσει ότι $S'$ είναι βέλτιστη λύση άρα και ελάχιστη.
Άρα: \\
\begin{center}
    $S > S' \Rightarrow 
    w_i \cdot \sum_1^i{p_k} + w_{i+1} \cdot \sum_1^{i+1}{p_k} > w_{i+1} \cdot \sum_1^{i-1}{p_k}
    + w_{i+1} \cdot p_{i+1} + w_{i} \cdot \sum_1^{i+i}{p_k}$ \\
    $ \Rightarrow w_i\cdot(p_1 + p_2 + \dots p_i) + w_{i+1}\cdot(p_1 + p_2 + \dots p_{i+1}) >
    w_{i+1}\cdot(p_1 + p_2 + \dots p_{i-1}) +
    w_{i}\cdot(p_1 + p_2 + \dots p_{i+1}) + w_{i+1} \cdot p_{i+1}$\\
    $\Rightarrow w_{i+1}\cdot p_i > w_i\cdot p_{i+1}$ \\
    $\text{  } $\\
  $\Rightarrow \dfrac{w_{i+1}}{p_{i+1}} > \dfrac{w_i}{p_i}$
\end{center}

Το οποίο σαφώς δεν ισχύει καθώς γνωρίζω ότι $\dfrac{w_{i}}{p_{i}} > \dfrac{w_{i+1}}{p_{i+1}}$,  $\forall
i \in A$. Άρα ο G(A) βρίσκει όντως βέλτιστη λύση. Η πολυπλοκότητα του συγκεκριμένου άπληστου
αλγορίθμου προκύπτει $O(n\cdot log(n))$ λόγω ταξινόμησης. \\

2. Πλέον η χρήση άπληστου αλγορίθμου δεν θα καταφέρει να λύσει το πρόβλημα, επομένως θα εργαστούμε
ξανά με δυναμικό προγραμματισμό. Ας δούμε αρχικά την περίπτωση δύο υπαλλήλων. Θα αποφασίζουμε κάθε
φορά ποιος υπάλληλος θα εξυπηρετήσει τον επόμενο πελάτη ελέγχοντας τις τιμές για τις δύο περιπτώσεις
και παίρνοντας κάθε φορά την μικρότερη τιμή, αφού πρώτα ταξινομήσουμε βέβαια τους πελάτες σύμφωνα
με τον λόγο $\dfrac{w}{p}$ σε φθήνουσα σειρά όπως είδαμε στο προηγούμενο ερώτημα.\\
\begin{enumerate}
    \item{Η περίπτωση να αναλάβει ο υπάλληλος 1 τον επόμενο πελάτη.}
    \item{Η περίπτωση να αναλάβει ο υπάλληλος 2 τον επόμενο πελάτη.}
\end{enumerate}

Έστω ότι ο υπάλληλος 1 θα εξυπηρετήσει ένα σύνολο $C_1$ πελατών. Αν οι πελάτες είναι στο σύνολο $C$
τότε ο υπάλληλος 2 θα εξυπηρετήσει το σύνολο $C_2 = C - C_1$. Αφού θα διατάξουμε τους πελάτες πριν
γίνει η εξυπηρέτηση μπορούμε να ελέγξουμε για οποιοδήποτε πιθανό χρόνο εξυπηρέτησης που θα κάνει 
ο υπάλληλος 1 $(p_{c_1})$ ποιος χρόνος θα προκύψει για τον υπάλληλο 2. \\

Επομένως προκύπτει το συμπέρασμα ότι χρειάζεται πάλι η χρήση μιας αναδρομικής συνάρτησης 
$A(n,p_{c_1})$ όπου στην εισαγωγή κάθε πελάτη σε κάποιο ταμείο να προστίθεται σε αυτή την
συνάρτηση στη θέση $i$ για το βήμα $i$ (δεδομένου των προηγούμενων εισαγωγών και ενός χρόνου $p_{c_1}$,
τον χρόνο δηλαδή που θα εξυπηρετεί ο Υπάλληλος 1) η βέλτιστη τιμή για εκείνο το βήμα. Η λύση με χρήση
της αναδρομικής αυτής σχέσης θα είναι για το $n$-ιοστό βήμα του αλγορίθμου για κάποιο χρόνο $p_{c_1}$
όπου προκύπτει η ελάχιστη τιμή για την συνάρτηση $Α$.\\ 
\pagebreak


Προκύπτει λοιπόν η εξής αναδρομική σχέση:\\
\begin{center}
    $A(i,p_{c_1}) =
    \begin{cases}
        0 & i = 0 \\
        min(A(i-1,p_{c_1}-p_i) + w_i\cdot p_{c_1},
            A(i-1,p_{c_1}) + w_i \cdot(\sum_{0}^{i}{p_k} - p_{c_1})) & i \neq 0
    \end{cases}$
\end{center}\\ 

\begin{center}
    Τελικά η λύση του αλγορίθμου μας θα είναι $\min\limits_{i}\{Α(n,p_{c_1|i})\}$ \\ 
\end{center}

Να σημειωθεί ότι $\forall j > i \Rightarrow A(i,p_j) = 0$ γιατί δεν θα είναι εφικτό σε ένα βήμα $i$ 
του αλγορίθμου να εισάγωγουμε έναν πελάτη $j$ καθώς ακολουθούμε την ταξινόμηση κατά $\dfrac{w}{p}$. 
Αφού η συνάρτηση αυτή χρησιμοποιεί μνήμη θα χρειαστεί μόνο μία φορά να διατρέξουμε κάθε πελάτη, και
μετά για κάθε πελάτη θα αυξόνται οι δυνατοί χρόνοι $p_{c_1}$ για τους οποίους θα πρέπει να βρούμε κάθε
δυνατή επιλογή. Η μνήμη του αλγορίθμου θα εξυπηρετήσει στο να μην κάνουμε πολλαπλές φορές αυτές τις πράξεις.
Συνολικά θα βρούμε $Subset\_Sum(p_i)$ πιθανούς χρόνους για το $p_{c_1}$. Η πολυπλοκότητα του αλγορίθμου λοιπόν
θα προκύψει $Ο(n\cdot Subset\_Sum(p_i))$. \\

Για $m$ υπαλλήλους θα χρειαστεί να ελέγξω όλους τους πιθανούς χρόνους για $m-1$ υπαλλήλους επομένως
η πολυπλοκότητα προκύπτει $O(n\cdot Subset\_Sum(p_i)^{m-1})$.


\pagebreak


\end{document}















